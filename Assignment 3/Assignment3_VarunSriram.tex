\documentclass[]{article}
\usepackage{lmodern}
\usepackage{amssymb,amsmath}
\usepackage{ifxetex,ifluatex}
\usepackage{fixltx2e} % provides \textsubscript
\ifnum 0\ifxetex 1\fi\ifluatex 1\fi=0 % if pdftex
  \usepackage[T1]{fontenc}
  \usepackage[utf8]{inputenc}
\else % if luatex or xelatex
  \ifxetex
    \usepackage{mathspec}
  \else
    \usepackage{fontspec}
  \fi
  \defaultfontfeatures{Ligatures=TeX,Scale=MatchLowercase}
\fi
% use upquote if available, for straight quotes in verbatim environments
\IfFileExists{upquote.sty}{\usepackage{upquote}}{}
% use microtype if available
\IfFileExists{microtype.sty}{%
\usepackage{microtype}
\UseMicrotypeSet[protrusion]{basicmath} % disable protrusion for tt fonts
}{}
\usepackage[margin=1in]{geometry}
\usepackage{hyperref}
\hypersetup{unicode=true,
            pdftitle={R Notebook},
            pdfborder={0 0 0},
            breaklinks=true}
\urlstyle{same}  % don't use monospace font for urls
\usepackage{color}
\usepackage{fancyvrb}
\newcommand{\VerbBar}{|}
\newcommand{\VERB}{\Verb[commandchars=\\\{\}]}
\DefineVerbatimEnvironment{Highlighting}{Verbatim}{commandchars=\\\{\}}
% Add ',fontsize=\small' for more characters per line
\usepackage{framed}
\definecolor{shadecolor}{RGB}{248,248,248}
\newenvironment{Shaded}{\begin{snugshade}}{\end{snugshade}}
\newcommand{\KeywordTok}[1]{\textcolor[rgb]{0.13,0.29,0.53}{\textbf{#1}}}
\newcommand{\DataTypeTok}[1]{\textcolor[rgb]{0.13,0.29,0.53}{#1}}
\newcommand{\DecValTok}[1]{\textcolor[rgb]{0.00,0.00,0.81}{#1}}
\newcommand{\BaseNTok}[1]{\textcolor[rgb]{0.00,0.00,0.81}{#1}}
\newcommand{\FloatTok}[1]{\textcolor[rgb]{0.00,0.00,0.81}{#1}}
\newcommand{\ConstantTok}[1]{\textcolor[rgb]{0.00,0.00,0.00}{#1}}
\newcommand{\CharTok}[1]{\textcolor[rgb]{0.31,0.60,0.02}{#1}}
\newcommand{\SpecialCharTok}[1]{\textcolor[rgb]{0.00,0.00,0.00}{#1}}
\newcommand{\StringTok}[1]{\textcolor[rgb]{0.31,0.60,0.02}{#1}}
\newcommand{\VerbatimStringTok}[1]{\textcolor[rgb]{0.31,0.60,0.02}{#1}}
\newcommand{\SpecialStringTok}[1]{\textcolor[rgb]{0.31,0.60,0.02}{#1}}
\newcommand{\ImportTok}[1]{#1}
\newcommand{\CommentTok}[1]{\textcolor[rgb]{0.56,0.35,0.01}{\textit{#1}}}
\newcommand{\DocumentationTok}[1]{\textcolor[rgb]{0.56,0.35,0.01}{\textbf{\textit{#1}}}}
\newcommand{\AnnotationTok}[1]{\textcolor[rgb]{0.56,0.35,0.01}{\textbf{\textit{#1}}}}
\newcommand{\CommentVarTok}[1]{\textcolor[rgb]{0.56,0.35,0.01}{\textbf{\textit{#1}}}}
\newcommand{\OtherTok}[1]{\textcolor[rgb]{0.56,0.35,0.01}{#1}}
\newcommand{\FunctionTok}[1]{\textcolor[rgb]{0.00,0.00,0.00}{#1}}
\newcommand{\VariableTok}[1]{\textcolor[rgb]{0.00,0.00,0.00}{#1}}
\newcommand{\ControlFlowTok}[1]{\textcolor[rgb]{0.13,0.29,0.53}{\textbf{#1}}}
\newcommand{\OperatorTok}[1]{\textcolor[rgb]{0.81,0.36,0.00}{\textbf{#1}}}
\newcommand{\BuiltInTok}[1]{#1}
\newcommand{\ExtensionTok}[1]{#1}
\newcommand{\PreprocessorTok}[1]{\textcolor[rgb]{0.56,0.35,0.01}{\textit{#1}}}
\newcommand{\AttributeTok}[1]{\textcolor[rgb]{0.77,0.63,0.00}{#1}}
\newcommand{\RegionMarkerTok}[1]{#1}
\newcommand{\InformationTok}[1]{\textcolor[rgb]{0.56,0.35,0.01}{\textbf{\textit{#1}}}}
\newcommand{\WarningTok}[1]{\textcolor[rgb]{0.56,0.35,0.01}{\textbf{\textit{#1}}}}
\newcommand{\AlertTok}[1]{\textcolor[rgb]{0.94,0.16,0.16}{#1}}
\newcommand{\ErrorTok}[1]{\textcolor[rgb]{0.64,0.00,0.00}{\textbf{#1}}}
\newcommand{\NormalTok}[1]{#1}
\usepackage{graphicx,grffile}
\makeatletter
\def\maxwidth{\ifdim\Gin@nat@width>\linewidth\linewidth\else\Gin@nat@width\fi}
\def\maxheight{\ifdim\Gin@nat@height>\textheight\textheight\else\Gin@nat@height\fi}
\makeatother
% Scale images if necessary, so that they will not overflow the page
% margins by default, and it is still possible to overwrite the defaults
% using explicit options in \includegraphics[width, height, ...]{}
\setkeys{Gin}{width=\maxwidth,height=\maxheight,keepaspectratio}
\IfFileExists{parskip.sty}{%
\usepackage{parskip}
}{% else
\setlength{\parindent}{0pt}
\setlength{\parskip}{6pt plus 2pt minus 1pt}
}
\setlength{\emergencystretch}{3em}  % prevent overfull lines
\providecommand{\tightlist}{%
  \setlength{\itemsep}{0pt}\setlength{\parskip}{0pt}}
\setcounter{secnumdepth}{0}
% Redefines (sub)paragraphs to behave more like sections
\ifx\paragraph\undefined\else
\let\oldparagraph\paragraph
\renewcommand{\paragraph}[1]{\oldparagraph{#1}\mbox{}}
\fi
\ifx\subparagraph\undefined\else
\let\oldsubparagraph\subparagraph
\renewcommand{\subparagraph}[1]{\oldsubparagraph{#1}\mbox{}}
\fi

%%% Use protect on footnotes to avoid problems with footnotes in titles
\let\rmarkdownfootnote\footnote%
\def\footnote{\protect\rmarkdownfootnote}

%%% Change title format to be more compact
\usepackage{titling}

% Create subtitle command for use in maketitle
\newcommand{\subtitle}[1]{
  \posttitle{
    \begin{center}\large#1\end{center}
    }
}

\setlength{\droptitle}{-2em}

  \title{R Notebook}
    \pretitle{\vspace{\droptitle}\centering\huge}
  \posttitle{\par}
    \author{}
    \preauthor{}\postauthor{}
    \date{}
    \predate{}\postdate{}
  

\begin{document}
\maketitle

\begin{Shaded}
\begin{Highlighting}[]
\CommentTok{# 1, 2 already done}
\end{Highlighting}
\end{Shaded}

\begin{Shaded}
\begin{Highlighting}[]
\CommentTok{#Queston 3}

\CommentTok{#Read the data from a cvs file, and convert every character vector to a factor wherever possible}
\NormalTok{cancer <-}\StringTok{ }\KeywordTok{read.csv}\NormalTok{(}\StringTok{"prostate_cancer.csv"}\NormalTok{, }\DataTypeTok{stringsAsFactors =} \OtherTok{FALSE}\NormalTok{)}

\CommentTok{#check if the data is properly imported and structured or not}
\KeywordTok{str}\NormalTok{(cancer)}
\end{Highlighting}
\end{Shaded}

\begin{verbatim}
## 'data.frame':    100 obs. of  10 variables:
##  $ id               : int  1 2 3 4 5 6 7 8 9 10 ...
##  $ diagnosis_result : chr  "M" "B" "M" "M" ...
##  $ radius           : int  23 9 21 14 9 25 16 15 19 25 ...
##  $ texture          : int  12 13 27 16 19 25 26 18 24 11 ...
##  $ perimeter        : int  151 133 130 78 135 83 120 90 88 84 ...
##  $ area             : int  954 1326 1203 386 1297 477 1040 578 520 476 ...
##  $ smoothness       : num  0.143 0.143 0.125 0.07 0.141 0.128 0.095 0.119 0.127 0.119 ...
##  $ compactness      : num  0.278 0.079 0.16 0.284 0.133 0.17 0.109 0.165 0.193 0.24 ...
##  $ symmetry         : num  0.242 0.181 0.207 0.26 0.181 0.209 0.179 0.22 0.235 0.203 ...
##  $ fractal_dimension: num  0.079 0.057 0.06 0.097 0.059 0.076 0.057 0.075 0.074 0.082 ...
\end{verbatim}

\begin{Shaded}
\begin{Highlighting}[]
\KeywordTok{head}\NormalTok{(cancer)}
\end{Highlighting}
\end{Shaded}

\begin{verbatim}
##   id diagnosis_result radius texture perimeter area smoothness compactness
## 1  1                M     23      12       151  954      0.143       0.278
## 2  2                B      9      13       133 1326      0.143       0.079
## 3  3                M     21      27       130 1203      0.125       0.160
## 4  4                M     14      16        78  386      0.070       0.284
## 5  5                M      9      19       135 1297      0.141       0.133
## 6  6                B     25      25        83  477      0.128       0.170
##   symmetry fractal_dimension
## 1    0.242             0.079
## 2    0.181             0.057
## 3    0.207             0.060
## 4    0.260             0.097
## 5    0.181             0.059
## 6    0.209             0.076
\end{verbatim}

\begin{Shaded}
\begin{Highlighting}[]
\CommentTok{#remove ID because it is an extra column to the data frame and is redundant}
\NormalTok{cancer <-}\StringTok{ }\NormalTok{cancer[}\OperatorTok{-}\DecValTok{1}\NormalTok{]}

\CommentTok{#check if the ID column is removed}
\KeywordTok{head}\NormalTok{(cancer)}
\end{Highlighting}
\end{Shaded}

\begin{verbatim}
##   diagnosis_result radius texture perimeter area smoothness compactness
## 1                M     23      12       151  954      0.143       0.278
## 2                B      9      13       133 1326      0.143       0.079
## 3                M     21      27       130 1203      0.125       0.160
## 4                M     14      16        78  386      0.070       0.284
## 5                M      9      19       135 1297      0.141       0.133
## 6                B     25      25        83  477      0.128       0.170
##   symmetry fractal_dimension
## 1    0.242             0.079
## 2    0.181             0.057
## 3    0.207             0.060
## 4    0.260             0.097
## 5    0.181             0.059
## 6    0.209             0.076
\end{verbatim}

\begin{Shaded}
\begin{Highlighting}[]
\CommentTok{#Get the number of patients in table}
\KeywordTok{table}\NormalTok{(cancer}\OperatorTok{$}\NormalTok{diagnosis_result)}
\end{Highlighting}
\end{Shaded}

\begin{verbatim}
## 
##  B  M 
## 38 62
\end{verbatim}

\begin{Shaded}
\begin{Highlighting}[]
\CommentTok{#Rename B to benign and M to malignant in the column diagnosis}
\NormalTok{cancer}\OperatorTok{$}\NormalTok{diagnosis <-}\StringTok{ }\KeywordTok{factor}\NormalTok{(cancer}\OperatorTok{$}\NormalTok{diagnosis_result, }\DataTypeTok{levels =} \KeywordTok{c}\NormalTok{(}\StringTok{"B"}\NormalTok{, }\StringTok{"M"}\NormalTok{), }\DataTypeTok{labels =} \KeywordTok{c}\NormalTok{(}\StringTok{"Benign"}\NormalTok{, }\StringTok{"Malignant"}\NormalTok{))}

\CommentTok{#Gives the result of B/ to all cases in percentage form rounded of to 1 decimal place}
\KeywordTok{round}\NormalTok{(}\KeywordTok{prop.table}\NormalTok{(}\KeywordTok{table}\NormalTok{(cancer}\OperatorTok{$}\NormalTok{diagnosis)) }\OperatorTok{*}\StringTok{ }\DecValTok{100}\NormalTok{, }\DataTypeTok{digits=}\DecValTok{1}\NormalTok{)}
\end{Highlighting}
\end{Shaded}

\begin{verbatim}
## 
##    Benign Malignant 
##        38        62
\end{verbatim}

\begin{Shaded}
\begin{Highlighting}[]
\CommentTok{#Normalizing is the process of bringing all values to a common scale. It's vital because the presence of values in various scales can hinder the process of kNN or data analysis in general and might yield wrong results}
\CommentTok{#Here we create a normalize funtion}
\NormalTok{normalize <-}\StringTok{ }\ControlFlowTok{function}\NormalTok{(x)}
\NormalTok{\{}
  \KeywordTok{return}\NormalTok{((x}\OperatorTok{-}\KeywordTok{min}\NormalTok{(x))}\OperatorTok{/}\NormalTok{(}\KeywordTok{max}\NormalTok{(x)}\OperatorTok{-}\KeywordTok{min}\NormalTok{(x)))}
\NormalTok{\}}


\CommentTok{#Getting a normailized data frame by passing cancer(data frame) values to the normalize function written above}
\NormalTok{cancer_normalized <-}\StringTok{ }\KeywordTok{as.data.frame}\NormalTok{(}\KeywordTok{lapply}\NormalTok{(cancer[}\DecValTok{2}\OperatorTok{:}\DecValTok{9}\NormalTok{], normalize))}

\CommentTok{#check if the function is normalized}
\KeywordTok{summary}\NormalTok{(cancer_normalized)}
\end{Highlighting}
\end{Shaded}

\begin{verbatim}
##      radius          texture         perimeter           area       
##  Min.   :0.0000   Min.   :0.0000   Min.   :0.0000   Min.   :0.0000  
##  1st Qu.:0.1875   1st Qu.:0.1875   1st Qu.:0.2542   1st Qu.:0.1639  
##  Median :0.5000   Median :0.4062   Median :0.3500   Median :0.2637  
##  Mean   :0.4906   Mean   :0.4519   Mean   :0.3732   Mean   :0.2989  
##  3rd Qu.:0.7500   3rd Qu.:0.7031   3rd Qu.:0.5188   3rd Qu.:0.4266  
##  Max.   :1.0000   Max.   :1.0000   Max.   :1.0000   Max.   :1.0000  
##    smoothness      compactness        symmetry      fractal_dimension
##  Min.   :0.0000   Min.   :0.0000   Min.   :0.0000   Min.   :0.0000   
##  1st Qu.:0.3219   1st Qu.:0.1384   1st Qu.:0.2189   1st Qu.:0.1364   
##  Median :0.4384   Median :0.2622   Median :0.3254   Median :0.2273   
##  Mean   :0.4484   Mean   :0.2889   Mean   :0.3442   Mean   :0.2657   
##  3rd Qu.:0.5753   3rd Qu.:0.3876   3rd Qu.:0.4379   3rd Qu.:0.3636   
##  Max.   :1.0000   Max.   :1.0000   Max.   :1.0000   Max.   :1.0000
\end{verbatim}

\begin{Shaded}
\begin{Highlighting}[]
\CommentTok{#Creating to data sets, one for training and other for testing. I chose 50-50 rows for each set for consistency}
\NormalTok{cancer_training <-}\StringTok{ }\NormalTok{cancer_normalized[}\DecValTok{1}\OperatorTok{:}\DecValTok{50}\NormalTok{,]}
\NormalTok{cancer_test <-}\StringTok{ }\NormalTok{cancer_normalized[}\DecValTok{51}\OperatorTok{:}\DecValTok{100}\NormalTok{,]}

\CommentTok{#Target variable which has not been included in our training and test data will be included}
\NormalTok{cancer_training_labels <-}\StringTok{ }\NormalTok{cancer[}\DecValTok{1}\OperatorTok{:}\DecValTok{50}\NormalTok{,}\DecValTok{1}\NormalTok{]}
\NormalTok{cancer_test_labels <-}\StringTok{ }\NormalTok{cancer[}\DecValTok{51}\OperatorTok{:}\DecValTok{100}\NormalTok{,}\DecValTok{1}\NormalTok{]}

\CommentTok{#kNN is part of package "class"}
\KeywordTok{library}\NormalTok{(class)}

\CommentTok{#using kNN() function to classify test data for various values of k}
\NormalTok{cancer_test_prediction <-}\StringTok{ }\KeywordTok{knn}\NormalTok{(}\DataTypeTok{train =}\NormalTok{ cancer_training, }\DataTypeTok{test=}\NormalTok{cancer_test, }\DataTypeTok{cl=}\NormalTok{cancer_test_labels, }\DataTypeTok{k=}\DecValTok{10}\NormalTok{)}
\NormalTok{cancer_test_prediction2 <-}\StringTok{ }\KeywordTok{knn}\NormalTok{(}\DataTypeTok{train =}\NormalTok{ cancer_training, }\DataTypeTok{test=}\NormalTok{cancer_test, }\DataTypeTok{cl=}\NormalTok{cancer_test_labels, }\DataTypeTok{k=}\DecValTok{9}\NormalTok{)}
\NormalTok{cancer_test_prediction3 <-}\StringTok{ }\KeywordTok{knn}\NormalTok{(}\DataTypeTok{train =}\NormalTok{ cancer_training, }\DataTypeTok{test=}\NormalTok{cancer_test, }\DataTypeTok{cl=}\NormalTok{cancer_test_labels, }\DataTypeTok{k=}\DecValTok{8}\NormalTok{)}
\NormalTok{cancer_test_prediction4 <-}\StringTok{ }\KeywordTok{knn}\NormalTok{(}\DataTypeTok{train =}\NormalTok{ cancer_training, }\DataTypeTok{test=}\NormalTok{cancer_test, }\DataTypeTok{cl=}\NormalTok{cancer_test_labels, }\DataTypeTok{k=}\DecValTok{7}\NormalTok{)}

\CommentTok{#The cross table shows the accuracy of all models (4 here).}
\KeywordTok{library}\NormalTok{(gmodels)}
\KeywordTok{CrossTable}\NormalTok{(}\DataTypeTok{x=}\NormalTok{cancer_test_labels, }\DataTypeTok{y =}\NormalTok{cancer_test_prediction, }\DataTypeTok{prop.chisq=}\OtherTok{FALSE}\NormalTok{)}
\end{Highlighting}
\end{Shaded}

\begin{verbatim}
## 
##  
##    Cell Contents
## |-------------------------|
## |                       N |
## |           N / Row Total |
## |           N / Col Total |
## |         N / Table Total |
## |-------------------------|
## 
##  
## Total Observations in Table:  50 
## 
##  
##                    | cancer_test_prediction 
## cancer_test_labels |         B |         M | Row Total | 
## -------------------|-----------|-----------|-----------|
##                  B |        26 |         2 |        28 | 
##                    |     0.929 |     0.071 |     0.560 | 
##                    |     0.619 |     0.250 |           | 
##                    |     0.520 |     0.040 |           | 
## -------------------|-----------|-----------|-----------|
##                  M |        16 |         6 |        22 | 
##                    |     0.727 |     0.273 |     0.440 | 
##                    |     0.381 |     0.750 |           | 
##                    |     0.320 |     0.120 |           | 
## -------------------|-----------|-----------|-----------|
##       Column Total |        42 |         8 |        50 | 
##                    |     0.840 |     0.160 |           | 
## -------------------|-----------|-----------|-----------|
## 
## 
\end{verbatim}

\begin{Shaded}
\begin{Highlighting}[]
\KeywordTok{CrossTable}\NormalTok{(}\DataTypeTok{x=}\NormalTok{cancer_test_labels, }\DataTypeTok{y =}\NormalTok{cancer_test_prediction2, }\DataTypeTok{prop.chisq=}\OtherTok{FALSE}\NormalTok{)}
\end{Highlighting}
\end{Shaded}

\begin{verbatim}
## 
##  
##    Cell Contents
## |-------------------------|
## |                       N |
## |           N / Row Total |
## |           N / Col Total |
## |         N / Table Total |
## |-------------------------|
## 
##  
## Total Observations in Table:  50 
## 
##  
##                    | cancer_test_prediction2 
## cancer_test_labels |         B |         M | Row Total | 
## -------------------|-----------|-----------|-----------|
##                  B |        26 |         2 |        28 | 
##                    |     0.929 |     0.071 |     0.560 | 
##                    |     0.667 |     0.182 |           | 
##                    |     0.520 |     0.040 |           | 
## -------------------|-----------|-----------|-----------|
##                  M |        13 |         9 |        22 | 
##                    |     0.591 |     0.409 |     0.440 | 
##                    |     0.333 |     0.818 |           | 
##                    |     0.260 |     0.180 |           | 
## -------------------|-----------|-----------|-----------|
##       Column Total |        39 |        11 |        50 | 
##                    |     0.780 |     0.220 |           | 
## -------------------|-----------|-----------|-----------|
## 
## 
\end{verbatim}

\begin{Shaded}
\begin{Highlighting}[]
\KeywordTok{CrossTable}\NormalTok{(}\DataTypeTok{x=}\NormalTok{cancer_test_labels, }\DataTypeTok{y =}\NormalTok{cancer_test_prediction3, }\DataTypeTok{prop.chisq=}\OtherTok{FALSE}\NormalTok{)}
\end{Highlighting}
\end{Shaded}

\begin{verbatim}
## 
##  
##    Cell Contents
## |-------------------------|
## |                       N |
## |           N / Row Total |
## |           N / Col Total |
## |         N / Table Total |
## |-------------------------|
## 
##  
## Total Observations in Table:  50 
## 
##  
##                    | cancer_test_prediction3 
## cancer_test_labels |         B |         M | Row Total | 
## -------------------|-----------|-----------|-----------|
##                  B |        25 |         3 |        28 | 
##                    |     0.893 |     0.107 |     0.560 | 
##                    |     0.568 |     0.500 |           | 
##                    |     0.500 |     0.060 |           | 
## -------------------|-----------|-----------|-----------|
##                  M |        19 |         3 |        22 | 
##                    |     0.864 |     0.136 |     0.440 | 
##                    |     0.432 |     0.500 |           | 
##                    |     0.380 |     0.060 |           | 
## -------------------|-----------|-----------|-----------|
##       Column Total |        44 |         6 |        50 | 
##                    |     0.880 |     0.120 |           | 
## -------------------|-----------|-----------|-----------|
## 
## 
\end{verbatim}

\begin{Shaded}
\begin{Highlighting}[]
\KeywordTok{CrossTable}\NormalTok{(}\DataTypeTok{x=}\NormalTok{cancer_test_labels, }\DataTypeTok{y =}\NormalTok{cancer_test_prediction4, }\DataTypeTok{prop.chisq=}\OtherTok{FALSE}\NormalTok{)}
\end{Highlighting}
\end{Shaded}

\begin{verbatim}
## 
##  
##    Cell Contents
## |-------------------------|
## |                       N |
## |           N / Row Total |
## |           N / Col Total |
## |         N / Table Total |
## |-------------------------|
## 
##  
## Total Observations in Table:  50 
## 
##  
##                    | cancer_test_prediction4 
## cancer_test_labels |         B |         M | Row Total | 
## -------------------|-----------|-----------|-----------|
##                  B |        25 |         3 |        28 | 
##                    |     0.893 |     0.107 |     0.560 | 
##                    |     0.595 |     0.375 |           | 
##                    |     0.500 |     0.060 |           | 
## -------------------|-----------|-----------|-----------|
##                  M |        17 |         5 |        22 | 
##                    |     0.773 |     0.227 |     0.440 | 
##                    |     0.405 |     0.625 |           | 
##                    |     0.340 |     0.100 |           | 
## -------------------|-----------|-----------|-----------|
##       Column Total |        42 |         8 |        50 | 
##                    |     0.840 |     0.160 |           | 
## -------------------|-----------|-----------|-----------|
## 
## 
\end{verbatim}

\begin{Shaded}
\begin{Highlighting}[]
\CommentTok{#We can conclude that k=7(cancer_test_prediction4) has the best accuracy.}
\end{Highlighting}
\end{Shaded}

\begin{Shaded}
\begin{Highlighting}[]
\CommentTok{#Question 4 }

\CommentTok{#importing library caret}
\KeywordTok{library}\NormalTok{(caret)}
\end{Highlighting}
\end{Shaded}

\begin{verbatim}
## Loading required package: lattice
\end{verbatim}

\begin{verbatim}
## Loading required package: ggplot2
\end{verbatim}

\begin{Shaded}
\begin{Highlighting}[]
\KeywordTok{library}\NormalTok{(ggplot2)}
\KeywordTok{library}\NormalTok{(lattice)}
\CommentTok{#Splitting data into training and test}
\CommentTok{#Using the matrix from the previous question}
\CommentTok{#using a seed with hard coded value to maintain consistency throughout the kNN process for multiple runs }
\KeywordTok{set.seed}\NormalTok{(}\DecValTok{2018}\NormalTok{)}

\CommentTok{#createDatapartition takes vector value, which takes 0.5 "50%" of the data rows and stores in index}
\NormalTok{index <-}\StringTok{ }\KeywordTok{createDataPartition}\NormalTok{(}\DataTypeTok{y=}\NormalTok{cancer}\OperatorTok{$}\NormalTok{diagnosis_result, }\DataTypeTok{p=}\FloatTok{0.5}\NormalTok{,}\DataTypeTok{list=}\OtherTok{FALSE}\NormalTok{)}

\CommentTok{#create training set using index}
\NormalTok{caret_training <-}\StringTok{ }\NormalTok{cancer[index,]}

\CommentTok{#create test data set using -index = other rows excluding index rows}
\NormalTok{caret_test <-}\StringTok{ }\NormalTok{cancer[}\OperatorTok{-}\NormalTok{index,]}

\CommentTok{#dim(caret_test);dim(caret_training);}
\CommentTok{#caret_test}

\CommentTok{#The original dataset has 2 values, Benign and Malignant, they should be considered categorical variables.To convert these to categorical variables, we can convert them to factors.}
\NormalTok{caret_training[[}\StringTok{"diagnosis"}\NormalTok{]]=}\KeywordTok{factor}\NormalTok{(caret_training[[}\StringTok{"diagnosis"}\NormalTok{]])}


\CommentTok{#Caret package provides train() function for training the data . Before the train() function is used, we implement trainControl() method first as it controls the computational nuances of the train() method.}

\CommentTok{#Below: The process of splitting the data into k-folds can be repeated a number of times, this is called Repeated k-fold Cross Validation. The final model accuracy is taken as the mean from the number of repeats. I feel that an iterative approach to find k is the best solution since kNN is simple and finding the right k requires trial and error.}

\NormalTok{training_control <-}\StringTok{ }\KeywordTok{trainControl}\NormalTok{(}\DataTypeTok{method =} \StringTok{"repeatedcv"}\NormalTok{, }\DataTypeTok{number=}\DecValTok{10}\NormalTok{, }\DataTypeTok{repeats =} \DecValTok{3}\NormalTok{)}

\KeywordTok{set.seed}\NormalTok{(}\DecValTok{2017}\NormalTok{)}

\CommentTok{#Doing the kNN process here, prepocessing the data in this scnario: "center"" sets mean value as 1 and "scale" sets Standard deviation to 1 . Tuning holds an integer value which checks the number of different k values to check.}
\NormalTok{knn_caret <-}\StringTok{ }\KeywordTok{train}\NormalTok{(diagnosis }\OperatorTok{~}\NormalTok{.,}\DataTypeTok{data=}\NormalTok{caret_training,}\DataTypeTok{method=}\StringTok{"knn"}\NormalTok{,}\DataTypeTok{trControl=}\NormalTok{training_control,}\DataTypeTok{preProcess=}\KeywordTok{c}\NormalTok{(}\StringTok{"center"}\NormalTok{,}\StringTok{"scale"}\NormalTok{),}\DataTypeTok{tuneLength=}\DecValTok{10}\NormalTok{ )}

\NormalTok{knn_caret}
\end{Highlighting}
\end{Shaded}

\begin{verbatim}
## k-Nearest Neighbors 
## 
## 50 samples
##  9 predictor
##  2 classes: 'Benign', 'Malignant' 
## 
## Pre-processing: centered (9), scaled (9) 
## Resampling: Cross-Validated (10 fold, repeated 3 times) 
## Summary of sample sizes: 45, 45, 45, 45, 46, 45, ... 
## Resampling results across tuning parameters:
## 
##   k   Accuracy   Kappa    
##    5  0.9727778  0.9252525
##    7  0.9727778  0.9220779
##    9  0.9527778  0.8766234
##   11  0.9472222  0.8575758
##   13  0.9327778  0.8281385
##   15  0.9272222  0.8168831
##   17  0.9205556  0.8040626
##   19  0.9138889  0.7865801
##   21  0.9072222  0.7683983
##   23  0.8872222  0.7259740
## 
## Accuracy was used to select the optimal model using the largest value.
## The final value used for the model was k = 7.
\end{verbatim}

\begin{Shaded}
\begin{Highlighting}[]
\NormalTok{caret_test_pred <-}\StringTok{ }\KeywordTok{predict}\NormalTok{(knn_caret, }\DataTypeTok{newdata =}\NormalTok{ caret_test)}
\NormalTok{caret_test_pred}
\end{Highlighting}
\end{Shaded}

\begin{verbatim}
##  [1] Malignant Benign    Malignant Malignant Malignant Malignant Malignant
##  [8] Malignant Malignant Malignant Benign    Malignant Malignant Malignant
## [15] Malignant Malignant Malignant Malignant Malignant Malignant Malignant
## [22] Malignant Malignant Benign    Malignant Benign    Benign    Malignant
## [29] Malignant Benign    Benign    Malignant Benign    Benign    Benign   
## [36] Malignant Benign    Malignant Benign    Malignant Benign    Malignant
## [43] Benign    Malignant Malignant Benign    Malignant Malignant Benign   
## [50] Benign   
## Levels: Benign Malignant
\end{verbatim}

\begin{Shaded}
\begin{Highlighting}[]
\CommentTok{#Checking the accuracy of k in both versions, caret package has an accuracy of 0.97 to 0.84(in the previous ques) when k=7 and the former has a better accuracy overall.}
\end{Highlighting}
\end{Shaded}

\begin{Shaded}
\begin{Highlighting}[]
\CommentTok{#Question 6}

\KeywordTok{confusionMatrix}\NormalTok{(caret_test_pred, caret_test}\OperatorTok{$}\NormalTok{diagnosis)}
\end{Highlighting}
\end{Shaded}

\begin{verbatim}
## Confusion Matrix and Statistics
## 
##            Reference
## Prediction  Benign Malignant
##   Benign        17         0
##   Malignant      2        31
##                                           
##                Accuracy : 0.96            
##                  95% CI : (0.8629, 0.9951)
##     No Information Rate : 0.62            
##     P-Value [Acc > NIR] : 2.048e-08       
##                                           
##                   Kappa : 0.9133          
##  Mcnemar's Test P-Value : 0.4795          
##                                           
##             Sensitivity : 0.8947          
##             Specificity : 1.0000          
##          Pos Pred Value : 1.0000          
##          Neg Pred Value : 0.9394          
##              Prevalence : 0.3800          
##          Detection Rate : 0.3400          
##    Detection Prevalence : 0.3400          
##       Balanced Accuracy : 0.9474          
##                                           
##        'Positive' Class : Benign          
## 
\end{verbatim}

\begin{Shaded}
\begin{Highlighting}[]
\CommentTok{#using the confusion matrix, our model is at an accuracy of 96%.}
\end{Highlighting}
\end{Shaded}


\end{document}
